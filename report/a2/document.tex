\documentclass{article} % Change to the appropriate document class, i.e. report, book, article...
\usepackage{style}

\title{CSE 272 assignment 2}
\author{Alisha Lawson\\Hallgeir Lien}
\date{}
\begin{document}

\maketitle
\newpage

\section*{Introduction}
In this assignment we explore different methods of estimating the irradiance over a line in a scene. The scene used for the assignment has a rectangular diffuse area light with corners $(-1.75,-1,-0.05)$ and $(1.75,-1,0.05)$ (area $A=0.35\ m^2$) and normal $\mathbf{n_l}=(0,1,0)$ and power $\phi=100\ W$. Above the area light there is a diffuse square with corners $(-1,0,-1)$ and $(1,0,1)$ with normal $\mathbf{n_s}=(0,1,0)$. Above the square there is a mirror with corners $(-2,1,-2)$ and $(2,1,2)$ and normal $(0,-1,0)$. We estimated the irradiance over a line $AB$ between the points $A=(-1,0,0)$ and $B=(1,0,0)$ using Monte Carlo path tracing, progressive photon mapping, adaptive(Metropolis) photon mapping and Metropolis path tracing. 

\section*{Task 1}
Our first task was to estimate the irradiance using path tracing. We divided the interval between $A$ and $B$ into 100 discrete points, and from each point we shot out 10 million rays using a distribution $p(x)=\cos theta / \pi$ where $\theta$ is the angle between the normal of the diffuse surface and the direction of the ray, giving a total of 1 billion rays. Since we use a diffuse square light, the irradiance estimate becomes
$$
F(x) \approx \frac{1}{N} \sum_{i=1}^N \frac{L(x)/\pi \cos \theta}{\cos \theta / \pi} = \frac{1}{N} \sum_{i=1}^N L(x)
$$

Figure \ref{fig:pathtracing} shows the irradiance over the interval for 1 million, 10 million, 100 million and 1 billion total sample rays. The measurement points are evenly distributed between the points $A$ and $B$, with point $0$ being at point $A$ and point $99$ being at point $B$.
\begin{figure}[h]
\subfigure[1 million sample rays]{
    \includegraphics[width=0.5\textwidth]{plots/pathtracing_irrad_1mill}\\
}
\subfigure[10 million sample rays]{
    \includegraphics[width=0.5\textwidth]{plots/pathtracing_irrad_10mill}\\
}
\subfigure[100 million sample rays]{
    \includegraphics[width=0.5\textwidth]{plots/pathtracing_irrad_100mill}\\
}
\subfigure[1 billion sample rays]{
    \includegraphics[width=0.5\textwidth]{plots/pathtracing_irrad_last}\\
}
    \caption{Irradiance estimate, path tracing}
\label{fig:pathtracing}
\end{figure}

As expected, this scene requires a lot of rays in order to minimize the noise. Even after 100 million rays (1 million rays per point) there is considerable noise. If we were actually rendering an image with a similar kind of scene where we have perhaps a million pixels instead of only 100, the rendering process would take an outrageous amount of time.

The irradiance estimate seems to converte to a value of $0.72W/m^2$ near the edges, and $0.3W/m^2$ near the middle of the interval.

\section*{Task 2}
Our second task was to render a scene using Adaptive Progressive Photon mapping. We rendered a 512x512 image where each pixel is represented by a measurement point. We utilized our ray tracer from the previous course CSE 168 to handle the scene rendering. The initial pass traces a ray sent from each image pixel into the scene. The measurement point is stored at the first hit diffuse surface. Specular surfaces do not store measurement points, rather the point continues along the path, reducing the pixel contribution by the specular reflectance. We converted the existing photon map kd-tree to hold the measurement points since we noticed severe performance limitations with the large amount of stored points. 

Adaptive photon mapping uses the adaptive Markov chain sampling to mutate the path and replica exchange to mix between uniform and Markov chain sampling to prevent becoming stuck any local peaks. During each successive photon pass, we first emit a photon from the square light in a randomly sampled direction, which is distributed according to $p(x)=\cos \theta/\pi$. This photon becomes the new good path if it hits a sample point. Otherwise we continue the pass by mutating the existing good path by the following:
$$du_i = sgn(2\xi_1 - 1)\xi_2^{(\frac{1}{d_i}+1)}$$
$$d_i = d_{i-1} + \gamma_i(a_i - a^*)$$
$$\gamma_i = \frac{1}{i}$$
$$a^* = 0.234$$

This mutation strategy is applied in four dimensions; the photon direction in spherical coordinates and the position along the square light. Additionally for this assignment, we store the spherical coordinates for multiple bounces to mutate. This photon then becomes the good path if it hits a measurement point. Otherwise we sample the existing good path again. The measurement point's flux and radius are adjusted every 10,000 iterations. We found that a lower number of photons per pass gave us a sharper image. Initially we set the measurement point radius to $0.25$. On each pass, we adjust the radius $ \hat{R}(x) $ and $\tau_N(x, \omega)$ according to the user-defined ratio $ \alpha $ set to 0.7. The only formula that differs from previous assignments is the flux correction premultiplying the BRDF $f_r(x,\omega,\omega_p)$ since we are now interested in the scene radiance:

$$ \tau_N(x, \omega) = \sum_{p=1}^{N(x)} f_r(x,\omega,\omega_p){\phi}'(x_p, \omega_p) $$

We normalize the radiance for each hit point by the probability of hitting a measurement point using a uniform photon.

$$ F = \frac{\tau_{N}(x, w)}{(\pi R(x)^2) N_{emitted}}*\frac{N_{hit}}{N_{total}} $$

Similar to Assignment 2, we adjusted the flux around the square edges. Initially, the scene corners were too bright since they sampled photons located within the radius. We added a normal check to ensure only the photons located on the same plane were accumulated. However, this resulted in very dark corners. We therefore decided to scale the contribution of photons that hit the boundary points whose radius extend beyond the diffuse squares. We calculated the scaling factor by utilizing the intersection points of the circle determined by the measurement point radius with each square's edges. The scaling factor is the ratio between the proportional area and the area assumed by the circle of radius $R(x)$. The scaling factor is applied to the newly accumulated flux $\tau_M$. While this approach is an approximation, we noticed an improvement in image quality above the previous two alternatives.

The results can be seen in figures \ref{fig:adaptive_100000}, \ref{fig:adaptive_1000000}, \ref{fig:adaptive_10000000} and \ref{fig:adaptiveppm_gray_100000000}.

\begin{figure}[H]
    \centering
    \includegraphics[width=0.8\textwidth]{imgs/adaptiveppm_gray_100000}\\
    \caption{Adaptive photon mapping, .1 million photons}
    \label{fig:adaptive_100000}
\end{figure}

\begin{figure}[H]
    \centering
    \includegraphics[width=0.8\textwidth]{imgs/adaptiveppm_gray_1000000}\\
    \caption{Adaptive photon mapping, 1 million photons}
    \label{fig:adaptive_1000000}
\end{figure}

\begin{figure}[H]
    \centering
    \includegraphics[width=0.8\textwidth]{imgs/adaptiveppm_gray_10000000}\\
    \caption{Adaptive photon mapping, 10 million photons}
    \label{fig:adaptive_10000000}
\end{figure}

\begin{figure}[H]
    \centering
    \includegraphics[width=0.8\textwidth]{imgs/adaptiveppm_gray_100000000}\\
    \caption{Adaptive photon mapping, 100 million photons}
    \label{fig:adaptive_10000000}
\end{figure}
\pagebreak

We finally rendered a pretty scene!

\subsection*{Error analysis}
We compare each pixel value after the last sample ray for Monte Carlo path tracing to each measurement point in order to visualize how the errors decreased over time. We calculated the mean square error over the interval for different number of pixels, $Err = 1/N_p \sum_{i=1}^{N_p} (F_{i,a} - F_{i,p})^2$, where $F_{i,a}$ and $F_{i,p}$ is the pixel value for point $i$ for specified intervals of Adaptive Photon Mapping for 100 million photons, and Monte Carlo path tracing for ~2.6 billion samples. Figure \ref{fig:adaptive_msq} shows the mean square error for different number of samples.

\begin{figure}[H]
    \centering
    \includegraphics[width=0.8\textwidth]{plots/adaptiveppm_gray_msq}\\
    \caption{Adaptive photon mapping, mean square error}
    \label{fig:adaptive_msq}
\end{figure}

Adaptive photon mapping converges quickly, but contains a larger mean squared error than the following Metropolis method. We tried a variety of debugging methods to identify the 0.02 discrepancy that met with little success. However, despite the small difference, we believe the image quality is comparible to Metropolis. 






\section*{Task 4 - Metropolis path tracing}
The last task was implementing a Metropolis light transport algorithm. A simplified view of this algorithm is that we first find a "good" path that gives a contribution. Then we slightly perturb this path, sample the new perturbed path, and either accept or reject this new path as the "good" one based on some acceptance probability $a$, and repeat. The theory is that if we do have a good path that gives a contribution $\mathbf{z}$, it's likely that a path $\mathbf{z}+\Delta \mathbf{z}$, that are perturbed by some amount $\Delta \mathbf{z}$, would also give a contribution, which allows us to more completely sample bright objects. This is often a problem for Monte Carlo path tracers because each pixel is evaluated independently which means that if there is a small bright object in the scene, one pixel may hit the object while neighboring pixels may miss it completely which introduces bright specles of noise in the image. 

Again we divide the interval into 100 discrete points. We then wish to estimate $E_i$, where $E_i$ is the irradiance for point $i$. The probability distribution of paths in Metropolis light transport is proportional to the intensities in the image, i.e. $p(\mathbf{z}) = I(\mathbf{z})/{\int I(\mathbf{z})d\mathbf{z}}$. We estimate the normalizing factor (the average intensity) 
$$
b=\int I(\mathbf{z})d\mathbf{z}\approx \frac{1}{N_s}\sum_{i=1}^{N_s} \frac{I(\mathbf{z}_i) \cos \theta_i}{\cos \theta_i/\pi} = \frac{1}{N_s}\sum_{i=1}^{N_s} I(\mathbf{z}_i)\pi
$$
by shooting $N_s$ rays from random positions with a distribution of $p = \cos \theta/\pi$. Now we use one of the paths from the seeding process as our first path $\mathbf{z}_0$. Each path $\mathbf{z}$ contibutes to one and only one point, by designating a interval of length $(1- (-1))/N_p=2/N_p$ to each point, where $N_p$ is the number of discrete points, over the full interval of length $2$. The contribution from each path is 
$$\Delta E = \frac{1}{M} \frac{I(\mathbf{z})N_p}{p(\mathbf{z})}a(\mathbf{z}_{old}\to \mathbf{z}_{new}) = \frac{1}{M} \frac{I(\mathbf{z})N_p}{I(\mathbf{z})/b} a(\mathbf{z}_{old}\to \mathbf{z}_{new}) = \frac{bN_p}{M} a(\mathbf{z}_{old}\to \mathbf{z}_{new})$$ 

where $M$ is the total number of samples. Figures \ref{fig:metropolis_irrad1}, \ref{fig:metropolis_irrad10} and \ref{fig:metropolis_irrad100} shows the irradiance estimates for the interval for 1 million, 10 million and 100 million sample rays, compared to Monte Carlo path tracing with the same number of total rays.

\begin{figure}[h]
    \centering
    \includegraphics[width=0.8\textwidth]{plots/metropolis_irrad_1mill}\\
    \caption{Metropolis vs. Monte Carlo, 1 million sample rays each}
    \label{fig:metropolis_irrad1}
\end{figure}

\begin{figure}
    \centering
    \includegraphics[width=0.8\textwidth]{plots/metropolis_irrad_10mill}\\
    \caption{Metropolis vs. Monte Carlo, 10 million sample rays each}
    \label{fig:metropolis_irrad10}
\end{figure}

\begin{figure}
    \centering
    \includegraphics[width=0.8\textwidth]{plots/metropolis_irrad_last}\\
    \caption{Metropolis, 100 million sample rays vs. Monte Carlo, 1 billion sample rays}
    \label{fig:metropolis_irrad100}
\end{figure}

We see that for this scene, the level of noise for Metropolis is significantly lower than that of Monte Carlo path tracing. Looking at the final distribution (figure \ref{fig:metropolis_irrad100}), we see that the amount of noise is about the same. However, keep in mind that the Monte Carlo used ten times as many rays to get there. The irradiance for the Metropolis run is slightly skewed (point 0 has a slightly higher irradiance than point 99); this is due to random variation. In this scene, there are two bright spots on the opposite sides of the interval, so we must rely on large steps in order to sample both sides. But even with large steps, there is a large probability that one side will be sampled more than the other.

\subsection*{Mutation strategy}
We implement the mutation strategy recommended in Csaba Kelemen et. al., "A Simple and Robust Mutation Strategy for the Metropolis Light Transport Algorithm". It's given in figure \ref{fig:mutation_metropolis}. As seen in the pseudo code, we mutate the angles by a smaller amount. This seems to work well with this scene, because when the path moves close to the middle, the solid angle of the reflected light source in the mirror is very small.

\begin{figure}[h]
\begin{algorithmic}
\STATE Let $\mathbf{u}$ be the old path.
\STATE Let $p_{large}$ be the probability of taking a large step.
\STATE Let $U_i$ be different random numbers between $0$ and $1$. 
\IF{$U_0 < p_large$}
    \STATE $\mathbf{u} \gets \left( 2U_1-1, 2\pi U_2, sin^{-1}(\sqrt{U_3})\right)$
\ELSE
    \STATE $du_1 \gets 1/64 e^{-\log(2048/64)U_1}$
    \STATE $du_2 \gets 1/64 e^{-\log(2048/64)U_2}\cdot 0.125$
    \STATE $du_3 \gets 1/64 e^{-\log(2048/64)U_3}\cdot 0.125$
    \IF{$U_4 < 0.5$}
        \STATE $\mathbf{u} \gets \mathbf{u} + (du_1, du_2, du_3)$
    \ELSE
        \STATE $\mathbf{u} \gets \mathbf{u} - (du_1, du_2, du_3)$
    \ENDIF
    \IF{$\mathbf{u}$ is outside the sample space}
        \STATE Add or subtract the maximum value of the invalid elements to those elements
    \ENDIF
\ENDIF
\end{algorithmic}
\caption{Mutation strategy used for our Metropolis implementation}
\label{fig:mutation_metropolis}
\end{figure}

\subsection*{Error analysis}
Next, we assumed that the irradiance for each discrete point after the last sample ray for Monte Carlo path tracing was the true irradiance, and used that to see how the error decreased over time. We calculated the mean square error over the interval for different number of sample rays, $Err = 1/N_p \sum_{i=1}^{N_p} (E_{i,m} - E_{i,p})^2$, where $E_{i,m}$ and $E_{i,p}$ is the irradiance for point $i$ for Metropolis at some number of samples, and Monte Carlo path tracing for 1 billion samples. Figure \ref{fig:metropolis_msq} shows the mean square error for different number of samples.

\begin{figure}
    \centering
    \includegraphics[width=0.8\textwidth]{plots/metropolis_msq}\\
    \caption{Metropolis, mean square error}
    \label{fig:metropolis_msq}
\end{figure}

Of course, this error might never go to zero for a few reasons. First, the solution has not completely converged for 1 billion rays of Monte Carlo, so eventually the Metropolis simulation will be "more correct" than the values we assume to be correct. Second, there is a slight error in the estimation of $b$. This is to be expected, since we shoot a finite number of seed rays.

We also plotted the absolute error for 1 million and 100 million samples. These plots can be seen in figures \ref{fig:metropolis_error1} and \ref{fig:metropolis_error100}. The plots also include the error of the Monte Carlo path tracing approach. We see that the Monte Carlo irradiance values have large peaks for some of the points, which is very typical for this sampling method.

\begin{figure}
    \centering
    \includegraphics[width=0.8\textwidth]{plots/metropolis_error_1mill}\\
    \caption{Absolute error, 1 million rays}
    \label{fig:metropolis_error1}
\end{figure}

\begin{figure}
    \centering
    \includegraphics[width=0.8\textwidth]{plots/metropolis_error_100mill}\\
    \caption{Absolute error, 100 million rays}
    \label{fig:metropolis_error100}
\end{figure}
\end{document}
