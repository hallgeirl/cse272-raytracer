\documentclass{article} % Change to the appropriate document class, i.e. report, book, article...
\usepackage{style}

\title{CSE272 assignment 1}
\author{Alisha Lawson\\Hallgeir Lien}
\date{}
\begin{document}

\maketitle
\newpage

\section{Introduction}
In this assignment we explore different methods of estimating the irradiance at a point in a scene. The scene used for the assignment has a square diffuse area light with corners $(-1,10,-1)$ and $(1,10,1)$ (area $A=4\ m^2$)and normal $(0,-1,0)$ and power $100\ W$, a mirror with corners $(5,4,-1)$ and $(5,6,1)$ and normal $(-1,0,0)$, and a plane with origin $(0,0,0)$ and normal $(0,1,0)$. We will estimate the irradiance at the point $A=(0,0,0)$ using path tracing, progressive photon mapping and a combination of path tracing and area light sampling with multiple importance sampling.

\section{Task 1}
Our first task was to estimate the irradiance using path tracing. We shot out up to 1 million rays from the point A, with a probability distribution of $p_P(x) = \cos \theta / \pi$, where $\theta$ is the angle between the normal at $A$ and the ray direction. The results are plotted in figure \ref{fig:pathtracing}. 
\begin{figure}[h]
\begin{tabular}{c}
\includegraphics[width=0.9\textwidth]{plots/irrad_pathtracing.png}\\
\end{tabular}
\caption{Irradiance estimate, path tracing}
\label{fig:pathtracing}
\end{figure}

We calculate the irradiance by summing over all the contributions and averaging:
$$F=\frac{1}{N}\sum_{i=1}^N \frac{f(x)}{p_P(x)} = \frac{\pi}{N}\sum_{i=1}^N \frac{f(x)}{\cos \theta_o}\ W/m^2$$
$f(x) = 0$ if the sample ray misses the light source, and $f(x) = (\Phi/(\pi\cdot A))\cdot \cos \theta_i$. Note that for this scene, the incident angle to the light's surface and outgoing angle from the point A is identical and cancels out, and that the power per area is $25\ W/m^2$, so the above estimate simplifies to 
$$F=\frac{\pi}{N}\sum_{i=1}^N 25\ W/m^2$$
The estimate seems to converge to a point near $F=0.39$ for the scene above. 

\section{Task 2}

\section{Task 3 - Multiple importance sampling}
In the third task we would estimate the irradiance by combining both area light sampling and path tracing using multiple importance sampling with the balance heuristic
$$
F = \frac{1}{N}\sum_{i=1}^n \sum_{j=1}^{n_i} \frac{f(x)}{\sum_k n_k/N \cdot p_k(x)}
$$
where $N$ is the total number of samples, $n_i$ is the number of samples taken with method $i$, $f(x)$ is the sample value and $p_k(x)$ is the probability of shooting a ray in the direction of $x$ using the probability distribution $p_k$. Figure \ref{fig:importance} shows the irradiance estimate as the number of samples increases. 

The area samples was gathered by choosing a random point on the area light surface with $p_A(x)=1/A$, then evaluating
$$
f(x) = \frac{25}{\pi} \cdot \frac{\cos^2 \theta}{||x-x^\prime||^2}
$$

The path tracing samples was gathered like in task 1.

\begin{figure}[h]
\begin{tabular}{c}
\includegraphics[width=0.9\textwidth]{plots/irrad_importance.png}\\
\end{tabular}
\caption{Irradiance estimate, importance sampling}
\label{fig:importance}
\end{figure}

As we see, even though the variance is quite high in the beginning, the estimate seems to drop to and fluctuate around $0.39$ after just a few samples, where the pure path tracing approach overestimated the irradiance for the first 200000 samples.

\end{document}
